\documentclass[a4paper,12pt]{report}
\usepackage{a4wide}
\usepackage[utf8]{inputenc}
\usepackage[T1]{fontenc}
\usepackage[french]{babel}
\usepackage{graphics}
\usepackage{graphicx}
\usepackage{amssymb}
\usepackage{amsmath}
\usepackage{amsmath}
\usepackage{theorem}
\usepackage{mathrsfs}
\usepackage[Dark,utopial]{quotchap}
\usepackage[toc,page]{appendix}
\renewcommand{\appendixtocname}{Annexe}
\renewcommand{\appendixname}{{\sffamily Annexe}}


\begin{document}

//La page de garde
\begin{center}
\vspace{\stretch{2}}
{\Large {\bf Université de Saint-Quentin-en-Yvelines}}\\
{\Large {\bf {Master 1: Calcul Haut Performance et Simulation }}}\\
\vspace{\stretch{1}}
 \textbf{{\Huge  \textit{Rapport de projet de la programmation numérique}}} \\
\vspace{\stretch{1}}
\hrule
\hrule
\vspace{\stretch{0.5}}
{\Huge \textbf{\textsc{ Réseau de Neurones à Convolution (reconaissance d’images) }}}\\
\vspace{\stretch{0.5}}
\hrule
\hrule
\vspace{\stretch{1}}
{\textbf{\textit {Préparé par:}}}\\
\vspace{\stretch{0.5}}
{\large\textsc{M$^{r}$.Khadimou Rassoul DIOP}}\\
{\large\textsc{M$^{r}$.Hery ANDRIANANTENAINA}}\\
{\large\textsc{M$^{r}$.Said TADJER }}\\
{\large\textsc{M$^{elle}$.Baya ABBACI}}\\
\vspace{\stretch{0.5}}
{\large
\begin{tabular}{ll}
M$^{r}$. \textsc{Mohammed Salah} IBNAMAR  & Encadreur    \\
\end{tabular}
}\\
\vspace{\stretch{0.5}}
{\Large\textbf{\textit{ Année 2019-2020}}}
\end{center}


//La table des matière
\tableofcontents


\chapter*{Introduction}
\addcontentsline{toc}{chapter}{Introduction}


\chapter{Définitions et Généralités}
\section{Réseau de Neurones convolutionnels}


\chapter{Pré-processing des images d'entrée}
\section{Conversion des images}
\section{Détection de bords}


\chapter{Entrainement du réseau de Neurones}
\section{Poids et fonction d'activation}
\section{Gradient de descente}


\chapter{Test du réseau de Neurones}
\section{Classification d'images}
\section{Evaluation des mauvaises classifications}


\chapter*{Conclusion}
\addcontentsline{toc}{chapter}{Conclusion}


\chapter*{Organisation du travail}
\addcontentsline{toc}{chapter}{Organisation du travail}
Le travail s’est fait essentiellement ensemble et de manière progressive.\\

Tout d’abord, notre encadreur M$^{r}$. \textsc{Mohammed Salah} IBNAMAR  nous a expliqué notre projet et ses objectifs et surtout son importance dans la vie active.\\

On se rencontrait des fois pour voir l'avancement  du travail et s'aider entre nous, envoyer notre travail a notre encadreur afin de l'améliorer en corrigeant nos erreurs. Et c'est dans ce cadre, que nous avons généré les différents codes:\\
\begin{itemize}
      \item [\star] Le code pour convertir les images en noir et blanc.
      \item [\star] Le code de la détection des bords des images.
      \item [\star] Le code de réseau de Neurones.\\
    \end{itemize}
    
Vers la fin, il a fallu se partager le travail afin d’être dans le temps. Certains se sont concentrés sur la recherche de la documentations, l'utilisation de Github, sur la programmation en C et d’autre sur Latex pour la rédaction du rapport.

\section{Les outils utilisés}
La programmation des différents codes pour la détection du cancer de notre projet est faite en langage C.\\

Pour la rédaction du rapport on s'est servit du Latex, qui est l'un des meilleur langage et un système de composition de documents.\\

On a également utiliser Github, la ou on a créé notre organisation qu'on a nommée: Artificial-Breast-Cancer-Detection, a partir de laquelle on a fait nos branche afin que chacun travaille de son coté et puisse avancer sans toucher a la branche principale qui est par défaut la branche master. 


\chapter*{Références bibliographiques}
\addcontentsline{toc}{chapter}{Références bibliographiques}


\end{document}

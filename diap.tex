\documentclass[9pt]{beamer}
\usepackage[utf8]{inputenc}
\usepackage[T1]{fontenc}
\usepackage{mathabx}
\usepackage{mathpazo}
\usepackage{eulervm}
\usepackage{natbib}

\setbeamercovered{transparent=20}
%% Load the markdown package
\usepackage[citations,footnotes,definitionLists,hashEnumerators,smartEllipses,tightLists=false,pipeTables,tableCaptions,hybrid]{markdown}
%%begin novalidate

\markdownSetup{rendererPrototypes={
 link = {\href{#2}{#1}},
 headingOne = {\section{#1}},
 headingTwo = {\subsection{#1}},
 headingThree = {\begin{frame}\frametitle{#1}},
 headingFour = {\begin{block}{#1}},
 horizontalRule = {\end{block}}
}}
%%end novalidate

\usetheme{Dresden}
\usefonttheme{serif}
\usecolortheme{rose}

\begin{center}
{\Large { Université de Saint-Quentin-en-Yvelines}} \\
{\Large {Master 1: Calcul Haut Performance et Simulation }}
\end{center}
\title{Réseau de Neurones à convolution \\ Reconnaissance d'images} \\
\author{\textsc{M$^{elle}$.Baya ABBACI} \\ \textsc{M$^{r}$.Hery ANDRIANANTENAINA} \\ \textsc{M$^{r}$.Khadimou Rassoul DIOP} \\ \textsc{M$^{r}$.Said TADJER }  }



\begin{document}

\maketitle


\begin{frame}
\renewcommand{\bibfont}{\footnotesize}
 \tableofcontents
 \frametitle{Plan de présentation}
\end{frame}



\begin{markdown}
%%begin novalidate

# Introduction

### Introduction
\begin{itemize}
    \item L'objectif des chercheurs était de construire une machine capable de reproduire certains aspects de l'intelligence humaine. \pause
    \item Les réseaux de neurones artificiels réalisés à partir du modèle biologique ne sont rien qu'une tentative de modélisation mathématique du cerveau humain.  \pause
    \item Ils sont donc conçus pour reproduire certaines de leurs caractéristiques comme: \pause
    \begin{itemize}
        \item [\star] La capacité d'apprentissage \pause
        \item [\star] La capacité de mémoriser l'information \pause
        \item [\star] La capacité de traiter des informations incomplètes \pause
    \end{itemize}
\end{itemize}

\end{frame}

%%%%%%%%%%%%%%%%%%%%%%


##Historique

###Historique

\begin{itemize}
    \item [\star] \textbf{Mac Culloch} et \textbf{Pitts} ont proposé dès 1943 les premiers neurones formels calqués sur les neurones biologiques. \pause
    \end{itemize}
    \begin{itemize}
    \item [\star] Les réseaux de neurones appelé « NNs» ont donc une histoire relativement jeune (60 ans) mais ce n'est que depuis le début des années 1990 que les applications intéressantes des réseaux de neurones ont vu le jour.
\end{itemize}

\end{frame}


%%%%%%%%%%%%%%%%%%%%%%%%%%

#Réseaux de Neurones

## Définitions
### Définitions

\begin{itemize}
    \item [\star] \textbf{Qu'est-ce qu'un neurone?} \pause
     \item Un neurone est une cellule d'un système permettant la communication et le traitement de l'information. \pause
     \end{itemize}
     \\
     \begin{itemize}
     \item [\star] \textbf{Qu'est-ce qu'un neurone artificiel ?} \pause
     \item Les réseaux de neurones artificiels sont un moyen de modéliser le mécanisme d'apprentissage et de traitement de l'information qui se produit dans le cerveau humain.
\end{itemize}

\end{frame}

%%%%%%%%%%%%%%%%%%%%%%%%


## Eléments de Neurone artificiel
### Eléments de Neurone artificiel

\begin{itemize}
    \item [\star] Les entrées "\textbf{X}" du neurone proviennent soit d’autres éléments "processeurs", soit de l’environnement. 
      \item [\star]  Les poids "\textbf{W}" déterminent l’influence de chaque entrée. 
       \item [\star] La fonction de combinaison "\textbf{b}" combine les entrées et les poids. 
       \item [\star] La fonction de transfert  calcule la sortie "Y" du neurone en fonction de la combinaison en entrée. 
     \end{itemize}
    
\begin{figure}[!htbp]
\begin{center}
\includegraphics[width=
9cm,height=2.5cm]{a.png}
\end{center}
\end{figure} 

\end{frame}

%%%%%%%%%%%%%%%%%%%%%%%%

## Configurations des réseaux de Neurones
### Configurations des réseaux de Neurones 


   \begin{itemize}
    \item [\star] \textbf{Réseaux partiellement
  connectés:}  Chaque neurone est relié à quelques neurones localisés dans son périmètre.
  \begin{figure}[!htbp]
\begin{center}
\includegraphics[width=
2.5cm,height=1.3cm]{1.png}
\end{center}
\end{figure} 
   \item [\star] \textbf{Réseaux à connexions  complètes:}  Chaque neurone est lié à tous les autres neurones du réseau.
   \begin{figure}[!htbp]
\begin{center}
\includegraphics[width=
2.5cm,height=1.3cm]{2.png}
\end{center}
\end{figure}
        \item [\star] \textbf{Réseaux à couches: }  Tous les neurones d'une couche sont connectés aux neurones de la couche en aval.
        \begin{figure}[!htbp]
\begin{center}
\includegraphics[width=
2.5cm,height=1.3cm]{3.png}
\end{center}
\end{figure}
     \end{itemize}    
     

      
      


\end{frame}


%%%%%%%%%%%%%%%%%%%%%%%%%

## Architecture des réseaux de Neurones 
### Architecture des réseaux de Neurones 

   \begin{itemize}
    \item [\star]  Les réseaux non bouclés 
     \end{itemize}    
     
   \begin{figure}[!htbp]
\begin{center}
\includegraphics[width=
6cm,height=2cm]{4.png}
\end{center}
\end{figure}

   \begin{itemize}
    \item [\star]  Les réseaux  bouclés 
     \end{itemize}    
     
\begin{figure}[!htbp]
\begin{center}
\includegraphics[width=
6cm,height=2cm]{5.png}
\end{center}
\end{figure}

\end{frame}


%%%%%%%%%%%%%%%%%%%%%%%%%%

## Réseau de neurones a convolution
### Réseau de neurones a convolution
  Le réseau de neurones à convolution (CNN) est un type de réseau de neurones artificiels acycliques (feed-forward), ils consistent en un empilage multicouche de perceptrons, Les réseaux neuronaux convolutifs ont de larges applications dans la reconnaissance d'image et vidéo.
    
   
   \begin{figure}[!htbp]
\begin{center}
\includegraphics[width=
9cm,height=3.5cm]{6.png}
\end{center}
\end{figure}

\end{frame}

%%%%%%%%%%%%%%%%%%%%%%%%%%


### 

 \begin{itemize}
     \item [\star] \textbf{Couches de correction (ReLU):} Pour améliorer l’efficacité du traitement on intercale entre les couches de traitement une
couche qui va opérer une fonction mReLU sur les signaux de sortie: $F (x) = max(0, x)$
 \end{itemize}
 
\begin{figure}[!htbp]
\begin{center}
\includegraphics[width=
5cm,height=2cm]{7.png}
\end{center}
\end{figure} 
\pause
 \begin{itemize}
     \item [\star] \textbf{Couche convolutif:} La couche de convolution est le bloc de construction de base d’un CNN. Trois paramètres
permettent de dimensionner le volume de la couche de convolution : la profondeur, le pas et
la marge. \pause
 \end{itemize}
 
\begin{itemize}
     \item [\star] \textbf{Couche de pooling (POOL) :} Le pooling est un autre concept trés important des CNNs, ce qui est une forme de sous-
échantillonnage de l’image. L’image d’entrée est découpée en une série de rectangles de n pixels
de côté ne se chevauchant pas (pooling).
 \end{itemize}
 
 

\end{frame}



%%%%%%%%%%%%%%%%%%%%%%%%%%

## Exemple 
### Exemple de traitement d'image
 
   
   \begin{figure}[!htbp]
\begin{center}
\includegraphics[width=
9cm,height=3.5cm]{p1.png}
\end{center}
\end{figure}

\end{frame}



%%%%%%%%%%%%%%%%%%%%%%%%%%


### Exemple de traitement d'image
 
   
   \begin{figure}[!htbp]
\begin{center}
\includegraphics[width=
9cm,height=3.5cm]{p2.png}
\end{center}
\end{figure}

\end{frame}

%%%%%%%%%%%%%%%%%%%%%%%%%%


### Exemple de traitement d'image
 
   
   \begin{figure}[!htbp]
\begin{center}
\includegraphics[width=
9cm,height=3.5cm]{p3.png}
\end{center}
\end{figure}

\end{frame}



%%%%%%%%%%%%%%%%%%%%%%%%%%


### Exemple de traitement d'image
 
   
   \begin{figure}[!htbp]
\begin{center}
\includegraphics[width=
9cm,height=3.5cm]{p5.png}
\end{center}
\end{figure}

\end{frame}



%%%%%%%%%%%%%%%%%%%%%%%%%%


### Exemple de traitement d'image
 
   
   \begin{figure}[!htbp]
\begin{center}
\includegraphics[width=
9cm,height=3.5cm]{p6.png}
\end{center}
\end{figure}

\end{frame}

%%%%%%%%%%%%%%%%%%%%%%%%%

#Pre-processing des images
##Conversion des images 
###Conversion des images:

   \begin{itemize}
\item [\star]  Notre source de donnée d’image sont tous en extension png mais nous avons besoin des images en extension pgm.
\item [\star] C'est pour ça que nous avons converti nos images sous terminal pour avoir l’extension pgm.
\item [\star] La conversion des images
en Grayscale permettra de bien appliquer les filtres aux images. Ainsi que bien détecter les contours sur l’image.
     \end{itemize}    
     \begin{figure}[!htbp]
\begin{center}
\includegraphics[width=
9.5cm,height=3cm]{11.png}
\end{center}
\end{figure}
   

\end{frame}


%%%%%%%%%%%%%%%%%%%%%%%%%
##Détection des bords
###Détection des bords

   \begin{itemize}
\item [\star]  Le but de l'opération est de transformer ces image en une autre de mêmes dimensions dans laquelle les contours apparaissent par convention en blanc sur fond noir.  \pause
\end{itemize}
\begin{itemize}
\item [\star] Les contours sont les lieux où on trouve les variations significatives de l'information. Pour la détection des bords, nous avons procédé à l'implémentation de fonctions de filtre.
     \end{itemize}    
    
\end{frame}


%%%%%%%%%%%%%%%%%%%%%%%%%

###Filtre de Sobel

   \begin{itemize}
\item [\star]   Le filtre de Sobel est un filtre qui faits des balayage triangulaire pour détecter les bords des images. Le résultat obtenu grâce à l’application de cette filtre est montre sur le slide.
     \end{itemize}    
    
\begin{figure}[!htbp]
\begin{center}
\includegraphics[width=1.5cm,height=1.5cm]{sans1.png}
\caption{Image après l'application du filtre de Sobel}
\end{center}
\end{figure}

\end{frame}

%%%%%%%%%%%%%%%%%%%%%%%%%

###Filtre de Kirsch

   \begin{itemize}
\item [\star]  Le filtre de kirsch est un détecteur de bord non linéaire qui trouve la force de bord maximale dans quelques directions prédéterminées. Mais il y aune différence entre ce filtre et de Sobel ou de Prewitt. 
\item [\star] La valeur du seuil est choisie empiriquement pour obtenir le meilleur compromis entre la suppression de bruit et la conservation des contours.
     \end{itemize}    
    
\begin{figure}[!htbp]
\begin{center}
\includegraphics[width=1.5cm,height=1.5cm]{sans3.png}
\caption{Image après l'application du filtre de Kirsch}
\end{center}
\end{figure}

\end{frame}


%%%%%%%%%%%%%%%%%%%%%%%%%

###Filtre de Prewitt

   \begin{itemize}
\item [\star]  Le filtre de Prewitt est à peu prêt comme le filtre Sobel mais la différence, c’est au niveau  du balayage, il fait une balayage rectangulaire. n pu obtenir une meilleur résultat  à partir du filtre de Prewitt
     \end{itemize}    
    
\begin{figure}[!htbp]
\begin{center}
\includegraphics[width=1.5cm,height=1.5cm]{sans4.png}
\caption{Image après l'application du filtre de Prewitt}
\end{center}
\end{figure}

\end{frame}

%%%%%%%%%%%%%%%%%%%%%%%%%

#Entrainement du réseau
###Entrainement du réseau de Neurones

   \begin{itemize}
    \item [\star]  BACK-PROPAGATION
    \item [\star]  POIDS ET FONCTION D’ACTIVATION
    \item [\star]  GRADIENT DE DESCENTE
     \end{itemize}    
    

\end{frame}


%%%%%%%%%%%%%%%%%%%%%%%%%



###Back-propagation
 
 \begin{itemize}
    \item [\star]  PROPAGATION DE LA PERTE TOTALE
    \item [\star]  MISE À JOUR DES POIDS
     \end{itemize}    
  

\end{frame}

%%%%%%%%%%%%%%%%%%%%%%%%%


###Fonctions d'activation
 
\begin{figure}[!htbp]
\begin{center}
\includegraphics[width=
7.5cm,height=4.5cm]{9.png}
\end{center}
\end{figure}
  

\end{frame}



%%%%%%%%%%%%%%%%%%%%%%%%%


###Gradient conjugué
 

   \begin{itemize}
    \item [\star]  MINIMISATION DE L’ERREUR
    \item [\star] PEU COÛTEUX ET DISPOSE DE BONNES PROPRIÉTÉS.
     \end{itemize}    

\end{frame}


%%%%%%%%%%%%%%%%%%%%%%%%%

#Tester le réseau

### Test du réseau de neurones

   \begin{itemize}
    \item [\star]  DEUX RÉSULTATS POSSIBLES
    \item [\star]  IMPLÉMENTATION
    \item [\star]  ÉVALUATION DES MAUVAISES CLASSIFICATIONS
     \end{itemize}    
    

\end{frame}



%%%%%%%%%%%%%%%%%%%%%%%%%


### Les sorties

   \begin{itemize}
    \item [\star]  DÉTÉCTION DE TUMEUR
    \item [\star]  NON DÉTÉCTION DE TUMEUR
     \end{itemize}    
    

\end{frame}

%%%%%%%%%%%%%%%%%%%%%%%%%


### Implémentation

   \begin{figure}[!htbp]
\begin{center}
\includegraphics[width=
7.5cm,height=4.5cm]{10.png}
\end{center}
\end{figure}
  
    

\end{frame}


%%%%%%%%%%%%%%%%%%%%%%%%%


### Evaluation des mauvaises classification

   \begin{itemize}
    \item [\star]  NOMBRE D’ERREURS A CHAQUE TEST
    \item [\star]  POURCENTAGE
     \end{itemize}    
    

\end{frame}


%%%%%%%%%%%%%%%%%%%%%%%%%
#Résultat et conclusion

### Résultat

   
\begin{figure}[!htbp]
\begin{center}
\includegraphics[width=
9cm,height=4.5cm]{r.png}
\end{center}
\end{figure}


\end{frame}







%%%%%%%%%%%%%%%%%%%%%%%%%%

%%novalidate
\end{markdown}

\begin{frame}
\renewcommand{\bibfont}{\footnotesize}
\frametitle{Bibliographie}

\bibliographystyle{apalike}
\bibliography{refs}
\begin{itemize}
    \item [] [1] https://fr.wikipedia.org/wiki/D\%C3\%A9tection\_de\_contours.\\
    \item [] [2]https://www.sciencedirect.com/science/article/pii/S2405959518304934.\\
    \end{itemize}
    \begin{itemize}
    
    \item [] [3] https://www.kaggle.com/paultimothymooney/breast- histopathology-images.
\end{itemize} 


\end{frame}


\end{document}
